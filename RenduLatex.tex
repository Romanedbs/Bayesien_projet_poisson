%%%%%%%%%%%%%%%%%%%%%%%%%%%%%%%%%%%%%%%%%%%%%%%%%%%%%%%%%%%%%%%%%%%%%%%%
%%                                                                    %%
%%       Vos modif sont à faire plus loin seulement !!                %%
%%                (voir à la ligne 68)                                %%
%%                                                                    %%
%%%%%%%%%%%%%%%%%%%%%%%%%%%%%%%%%%%%%%%%%%%%%%%%%%%%%%%%%%%%%%%%%%%%%%%%


\documentclass[a4paper,11pt,twoside,openany]{report}
\usepackage{fullpage}
\usepackage[utf8]{inputenc}
\usepackage[frenchb]{babel}
\usepackage{amsmath,amssymb,amsthm}
\usepackage{graphicx}
\usepackage[labelfont=bf,textfont=sl,tableposition=top,small]{caption}
\usepackage{fancyhdr}
\usepackage{calc}
\usepackage[square]{natbib}

%%Pour avoir des pages réellement vides entre les chapitres
% \makeatletter
% \def\cleardoublepage{\clearpage\if@twoside \ifodd\c@page\else
% \hbox{}
% \vspace*{\fill}
% \vspace{\fill}
% \thispagestyle{empty}
% \newpage
% \if@twocolumn\hbox{}\newpage\fi\fi\fi}
% \makeatother

%% En-tête et pieds de page avec fancyhdr
\setlength{\headsep}{13.6pt}
\headheight=14.85pt
\renewcommand{\chaptermark}[1]{\markboth{\thechapter.\ #1}{}}
\renewcommand{\sectionmark}[1]{\markright{\thesection\ #1}}
\fancyhf{}
\fancyhead[RO]{\bfseries\rightmark}
\fancyhead[LE]{\bfseries\leftmark}
\fancyfoot[LE,RO]{\bfseries\thepage}
\renewcommand{\headrulewidth}{0pt}
\addtolength{\headheight}{0.5pt}
 \renewcommand{\footrulewidth}{0pt}
\fancypagestyle{plain}{\fancyhead{}\renewcommand{\headrulewidth}{0pt}}

%% Mise en page des théorèmes, lemme etc...
\theoremstyle{plain}
\newtheorem{thm}{Théorème}[section]
\newtheorem{lemme}[thm]{Lemme}
\newtheorem{prop}[thm]{Proposition}
\newtheorem*{cor}{Corollaire}
\theoremstyle{definition}
\newtheorem{defn}{Définition}[section]
\newtheorem{conj}{Conjecture}[section]
\newtheorem{exmp}{Exemple}[section]
\theoremstyle{remark}
\newtheorem*{rem}{Remarque}
\newtheorem*{note}{Note}
\newtheorem{case}{Cas particulier}

\renewcommand{\floatpagefraction}{0.95}
\renewcommand{\textfraction}{0.05}

\newcommand{\HRule}{\rule{\linewidth}{0.5mm}}

%% Chemin du repertoire contenant toutes les Figures
\graphicspath{{./Figures/}}

%%%%%%%%%%%%%%%%%%%%%%%%%%%%%%%%%%%%%%%%%%%%%%%%%%%%%%%%%%%%%%%%%%%%%%%%
%%                                                                    %%
%%       Faire vos modifications à partir d'ici !!!!                  %%
%%                                                                    %%
%%%%%%%%%%%%%%%%%%%%%%%%%%%%%%%%%%%%%%%%%%%%%%%%%%%%%%%%%%%%%%%%%%%%%%%%

%% La ligne suivante est a commenter pour enlever le Lorem Ipsum !!!!
%% Attention il faudra aussi supprimer toutes les commandes
%% \Blindtext, \blindtext, \blindmathpaper
%%\usepackage{blindtext}

\title{Le Titre de mon super }
\author{Monaury Clément \& Dubois Romane}
\date{22 décembre 2023}

\begin{document}

\pagestyle{empty}
\begin{titlepage}
  \begin{center}
    \vspace*{-2cm}
    % \includegraphics[width=0.25\textwidth]{logoUM2}\\[1cm]

    \textsc{\LARGE Université Rennes}\\[1.5cm]

    \HRule\\[0.4cm]

    {\huge \textbf{Modèles à espaces d'états intégrés : La pêcherie de merlu en Namibie}}\\[0.4cm]
    
    \HRule\\[1.5cm]

    Compte rendu de TP -  UE EPB
    Master 2 MODE
    % \includegraphics[width=\textwidth]{figurePage2garde}
    \vfill
    \begin{minipage}{0.4\textwidth}
      \begin{flushleft}
        \large
        Clément \textsc{Monaury}\\
        Romane \textsc{Dubois}
      \end{flushleft}
    \end{minipage}
    \begin{minipage}{0.4\textwidth}
      \begin{flushright}
        \large
        22 décembre 2023\\
      \end{flushright}
    \end{minipage}

    \vfill
  
  \end{center}
\end{titlepage}

\pagenumbering{roman}


\pagestyle{fancy}
%\setcounter{page}{0}
\pagenumbering{arabic}

\section*{Introduction}
  Les problématiques de surpèche et perte de biodiversité dans nos océans sont loins d'être un problème récent. 
*petite phrase sur la diminution de la biomasse de poisson* .
De nombreuses études ont été menées (ref) afin de modéliser et prédire au mieux la biomasse des différentes 
espèces concernées. *blabla*

Une des méthodes qui donnent des résultats intéressant est l'utilisation de modèle à espaces d'états 
intégrés développé durant ce TP. C'est à dire des modèles qui qui rentre dans l'idée de modélisation 
hiérarchique avec une composante dynamique dans la couche latente de la structure hiérarchique. 
Et modèle "intégrés" car on combine plusieurs jeux de donnés de sources différentes, 
cela permet de diversifier les informations et d'estimer au mieux les paramètres par rapport à la réalité.

Ce rapport s'appuie sur *blabla*

Dans un premier temps nous ferons la comparaison d'un premier modèle hiérarchique avec comme modèle 
de production de biomasse le modèle de Shaefer et un deuxième avec le modèle de Fox. Après analyse des 
résultats de comparaison, nous ferrons dans un second temps une analyse de prédiction grâce au modèle 
choisi, pour prédire l'évolution de la biomasse du stocke de merlu en fonction de différents scénarios 
de pêches. L'étape de modélisation utltérieur à la prédiction nous permet de prendre en compte et de 
distinguer l'aléa du processus et l'aléa de l'observation dans la prédiction de la biomasse.


\section*{Présentation des modèles testés}

=> Schéma du modèle hiérarchique global 

=> Paramètre inconnu


=> Modèle de processus 

Les modèles de processus ici utilisés sont des modèles de production de biomasse qui vont 
nous servir à étudier l'évolution de la biomasse et le rendement équilibré maximal (MSY). 
Nous avons choisis de comparer deux modèles. Les deux se compose d'une fonction de production $g(B)$ 
au quel on soustrait les prises aux temps $t$ $C_t$:
$$B_{t+1} =  g(B_t)-C_t$$
On ne représente ni la strucuture d'âge, de taille ou de sexe.

Les modèles testés diffèrent dans la manière de définir la fonction de production $g(B_t)$. 

- Le modèle de Shaefer :
$$\dfrac{dB_t}{dt}= r*B_t*(1 - \dfrac{B_t}{K}) $$
- Le modèle de Fox :

Avex $r$ la croissance de la population et $K$ la capacité maximale de population. Si on arrive à avoir $r$ 
et $K$ on peut avoir accès à des point de références sur la gestion telle que le MSY qui correspond à la 
prise maximale ne comprommettant pas la survie à long terme de la population (C.Bordet,2013).

On calcul le $C_{MSY}$ de deux manières différentes en fonction des modèles testés. 
- Le modèle de Shaefer :
$$C_{MSY}=\dfrac{R*K}{4} $$
- Le modèle de Fox : 

Ces deux modèles sont en faite deux cas particulier du modèle de Pella et Tomlinson (Pella and Tomlinson, 1969):
(formule)

avec $m$ un paramètre d'asymétrie.

*Rajouter quelques mots sur ces modèles ?*

=> intégration de deux de donnés 
%% Voici comment insérer une figure !!! Utiliser de préférence le
%% format eps !!!!
% \begin{figure}
%   \centering
%   \includegraphics[width=\textwidth]{studyRegionApp}
%   \caption[Titre plus synthétique aller voir comme il diffère du titre
%   original.]{Left: Topographical map of Switzerland showing the sites
%     and altitudes in metres above sea level of 16 weather stations for
%     which annual maxima temperature data are available. Middle: Times
%     series of the daily maxima temperatures at the 16 weather stations
%     for year 2003. The '$\rm o$', '$+$' and '$\rm x$' symbols indicate
%     the annual maxima that occurred the 11th, 12th and 13th of August
%     respectively. Right: Comparison between the fitted extremal
%     coefficient function from a Schlather process (solid red line) and
%     the pairwise extremal coefficient estimates (gray circles). The
%     black circles denote binned estimates with 16 bins.}
%   \label{fig:studyRegion}
% \end{figure}



% \chapter{Un chapitre}
% \label{cha:un-chapitre}

% \section{Une autre section}
% \label{sec:une-autre-section}


% %% Voici comment insérer un tableau !!! Attention pas de ligne verticale
% %% !!!
% \begin{table}
%   \centering
%   \caption[Spatial dependence structures of Schlather processes based
%   on an isotropic powered exponential correlation function]{Spatial
%     dependence structures of Schlather processes based on an isotropic
%     powered exponential correlation function with scale parameter
%     $\lambda$ and shape parameter $\kappa$. The correlation function
%     parameters are set to ensure that $\theta(100) = 1.5$.}
%   \label{tab:spatDepConf}
%   \begin{tabular}{lccc}
%     \hline
%     & \multicolumn{3}{c}{Sample path properties}\\
%     & $\theta_1\colon \text{Wiggly}$ & $\theta_2\colon \text{Smooth}$
%     & $\theta_3\colon \text{Very smooth}$\\
%     \hline
%     $\lambda$ & 208 & 144 & 128\\
%     $\kappa$ & 0.5 & 1 & 1.5\\
%     \hline
%   \end{tabular}
% \end{table}


% \subsection{Une sous-section}
% \label{sec:une-sous-section}



% \subsection{Une autre sous-section}
% \label{sec:une-autre-sous-1}



% \begin{figure}
%    \centering
%    \includegraphics[width=\textwidth]{condSimApp3}
%    \caption[Maps on a $64 \times 64$ grid of the pointwise $0.025,
%    0.5$ and $0.975$ sample quantiles for temperature $(^\circ{\rm C})$
%    obtained from 10000 conditional simulations of the fitted Schlather
%    process.]{From left to right: Maps on a $64 \times 64$ grid of the
%      pointwise $0.025, 0.5$ and $0.975$ sample quantiles for
%      temperature $(^\circ{\rm C})$ obtained from 10000 conditional
%      simulations of the fitted Schlather process. The squares show the
%      conditional locations and the conditional values. The right panel
%      shows the temperatures anomalies, i.e., the difference between
%      the pointwise conditional medians and the pointwise unconditional
%      medians estimated from the fitted trend.}
%   \label{fig:condSimApp}
% \end{figure}



% \section{Encore une section}
% \label{sec:encore-une-section}



% \chapter{Un autre chapitre}
% \label{cha:un-autre-chapitre}


% \chapter*{Conclusion}




% \nocite{R,Dav,Agr,nat,Li:1985,BrCl,CM,LesSpi,LitRub,MCN}
% \bibliographystyle{apalike}  
% \bibliography{references}

% \appendix

% \chapter{Une appendice}
% \label{cha:une-appendice}



% \chapter{Une deuxième appendice}
% \label{cha:une-deux-append}



\end{document}
